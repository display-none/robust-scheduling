\documentclass{article} 
\usepackage[T1]{fontenc}
\usepackage[english]{babel}
\usepackage[utf8]{inputenc}
\usepackage{lmodern}
\usepackage{mathtools}
\usepackage{longtable}
\usepackage{graphicx}
\usepackage{algpseudocode}
\usepackage{amsmath}
\usepackage{amsthm}

\theoremstyle{definition}
\newtheorem{definition}{Def}[section]

\newtheorem{theorem}{Tw}[section]

%%% fix for \lll
\let\babellll\lll
\let\lll\relax

\usepackage{amssymb}

%%% fix for \lll
\let\mathlll\lll
\let\lll\babellll


\usepackage{pdflscape}
\providecommand{\e}[1]{\ensuremath{\times 10^{#1}}}

\begin{document} 

\title{Robust constraint-based scheduling with activity duration scenario sets}
\author{Jacek Hoła}
\date{\today}
\maketitle
\newpage

\tableofcontents
\newpage

\section{Introduction}

\section{Main problem and related concepts}
In this section we will introduce the problem we will be considering and concepts directly related to its formulation.


\subsection{General principles of scheduling}
Scheduling can most generally be defined as a problem of allocating scarce resources to activities over time \cite{Baker:Introduction}. 
Some additional limitations can be added to the problem. For example, some activities are ready at some specified time and cannot be scheduled earlier, some activities must precede other etc.

Among scheduling problems we distinguish \textit{non-preemptive}, \textit{preemptive} and \textit{elastic} problems. In non-preemptive scheduling, activities cannot be interrupted. In preemptive scheduling, an activity can be executed in time periods, being interrupted at any time. In elastic scheduling, one can assign some amount of resource to the activity (between 0 and resource capacity) and execute the activity in any periods, as long as the total amount of the resource assigned to the activity over time is equal to a specified value called \textit{energy}.

Scheduling problems can be solved as \textit{decision} problems, \textit{function} problems or \textit{optimization} problems. When considering a decision problem, one has to answer a question "Does there exist a schedule satisfying all the constraints?". For a function problem the goal is to provide a valid schedule. Optimization problems need to have an objective function defined for them. The goal is then to find a valid schedule, for which the value of the objective function is minimal or maximal, depending on the function. The most common objective is minimization of makespan (i.e. the time that elapses between the start time of the first activity and the end time of the last activity in the schedule).

\subsection{Resource-Constrained Project Scheduling Problem}
The problem we are considering is based on Resource-Constrained Project Scheduling Problem (RCPSP), which is a fairly generic framework to describe scheduling problems encountered in industry and academic discussions. It consists of a set of $n$ \textit{activities} $\{A_1, A_2, ..., A_n\}$ with fixed processing times $proc(A_i)$ or $p_i$ (activity duration) and a set of $m$ resources $\{R_1, R_2, ..., R_m\}$ of specified capacity $cap(R_i)$. Each activity $A_i$ requires $cap(A_i, R)$ of resource $R$ during its execution and at any given time the sum of the requirements from all executing activities cannot exceed the resource capacity. Formally, for all times t:

$$
\sum_{A_i : start(A_i) \leq t < end(A_i)} cap(A_i, R) \leq cap(R)
$$

$start(A_i)$ and $end(A_i)$ used here are variables describing the start time and end time of the activity $A_i$, respectively. When activities cannot be interrupted (are non-preemptive) it follows that $end(A_i) = start(A_i) + proc(A_i)$. We would only consider non-preemptive problems, so to define a schedule it suffices to produce a vector of start times $\langle start(A_1), ..., start(A_n)\rangle$.

In addition to mentioned constraints the activities can have release dates, due dates, deadlines and be subject to precedence constraints. A release date is a time before which the activity cannot execute. A due date is a time before which the activity is expected to end and similarly, a deadline is a time before which the activity has to end. Precedence constraints can be defined in several ways. An activity can be constrained to start after the start of some other activity, start after the end of some other activity or some defined time after either of these.


\subsection{Graham's notation}

\subsection{Problem formulation}



\section{Robustness and uncertainty}

\subsection{In general}

\subsection{In scheduling}

\subsection{In our problem}




\section{Solution - Constraint Programming}

\subsection{What is Constraint Programming and how does it apply to scheduling?}

\subsection{Constraint propagation algorithms for scheduling}

\subsection{Precedence Constraint Posting}

\subsection{Proposed solution}


\section{Computational results}

\subsection{Experiment description}

\subsection{Problem instances}

\subsection{Results}


\section{Conclusions}



\bibliographystyle{plain}
\bibliography{thesis}


\end{document}