\documentclass{article} 
\usepackage[T1]{fontenc}
\usepackage[english]{babel}
\usepackage[utf8]{inputenc}
\usepackage{lmodern}
\usepackage{mathtools}
\usepackage{longtable}
\usepackage{graphicx}
\usepackage{algpseudocode}
\usepackage{amsmath}
\usepackage{amsthm}

\theoremstyle{definition}
\newtheorem{definition}{Def}[section]

\newtheorem{theorem}{Tw}[section]

%%% fix for \lll
\let\babellll\lll
\let\lll\relax

\usepackage{amssymb}

%%% fix for \lll
\let\mathlll\lll
\let\lll\babellll


\usepackage{pdflscape}
\providecommand{\e}[1]{\ensuremath{\times 10^{#1}}}

\begin{document} 

\title{Robust constraint-based scheduling with activity duration scenario sets}
\author{Jacek Hoła}
\date{\today}
\maketitle
\newpage

\tableofcontents
\newpage

\section{Introduction}

\section{Main problem and related concepts}
In this section we will introduce the problem we will be considering and concepts directly related to its formulation. Firstly, we'll introduce main ideas of scheduling, next its general formulation, popular in the field - Resource-constrained Project Scheduling Problem. Then we will quickly explain the notation used to describe scheduling problems - the Graham's notation. Finally, we'll try to formulate the problem of the thesis.


\subsection{General principles of scheduling}
Scheduling can most generally be defined as a problem of allocating scarce resources to activities over time \cite{Baker:Introduction}. 
Some additional limitations can be added to the problem. For example, some activities are ready at some specified time and cannot be scheduled earlier, some activities must precede other etc.

Among scheduling problems we distinguish \textit{non-preemptive}, \textit{preemptive} and \textit{elastic} problems. In non-preemptive scheduling, activities cannot be interrupted. In preemptive scheduling, an activity can be executed in time periods, being interrupted at any time. In elastic scheduling, one can assign some amount of resource to the activity (between 0 and resource capacity) and execute the activity in any periods, as long as the total amount of the resource assigned to the activity over time is equal to a specified value called \textit{energy}.

Scheduling problems can be solved as \textit{decision} problems, \textit{function} problems or \textit{optimization} problems. When considering a decision problem, one has to answer a question "Does there exist a schedule satisfying all the constraints?". For a function problem the goal is to provide a valid schedule. Optimization problems need to have an objective function defined for them. The goal is then to find a valid schedule, for which the value of the objective function is minimal or maximal, depending on the function. The most common objective is minimization of makespan (i.e. the time that elapses between the start time of the first activity and the end time of the last activity in the schedule).

\subsection{Resource-Constrained Project Scheduling Problem}
The problem we are considering is based on Resource-Constrained Project Scheduling Problem (RCPSP), which is a fairly generic framework to describe scheduling problems encountered in industry and academic discussions. It consists of a set of $n$ \textit{activities} $\{A_1, A_2, ..., A_n\}$ with fixed processing times $proc(A_i)$ or $p_i$ (activity duration) and a set of $m$ resources $\{R_1, R_2, ..., R_m\}$ of specified capacity $cap(R_i)$. Each activity $A_i$ requires $cap(A_i, R)$ of resource $R$ during its execution and at any given time the sum of the requirements from all executing activities cannot exceed the resource capacity. Formally, for all times t:

$$
\sum_{A_i : start(A_i) \leq t < end(A_i)} cap(A_i, R) \leq cap(R)
$$

$start(A_i)$ and $end(A_i)$ used here are variables describing the start time and end time of the activity $A_i$, respectively. When activities cannot be interrupted (are non-preemptive) it follows that $end(A_i) = start(A_i) + proc(A_i)$. We would only consider non-preemptive problems, so to define a schedule it suffices to produce a vector of start times $\langle start(A_1), ..., start(A_n)\rangle$.

In addition to mentioned constraints the activities can have release dates, due dates, deadlines and be subject to precedence constraints. A release date is a time before which the activity cannot execute. A due date is a time before which the activity is expected to end and similarly, a deadline is a time before which the activity has to end. Precedence constraints can be defined in several ways. An activity can be constrained to start after the start of some other activity, start after the end of some other activity or some defined time after either of these.


\subsection{Graham's notation}
// shortly describe what is graham's notation

\subsection{Problem formulation}
The problem we will tackle in this thesis is a variation of RCPSP. Using Graham's notation we can classify it as $P_m | r_i, prec | C_{max}$. That means we have $m$ identical machines and our goal is to schedule some number of activities, each having a release date and possibly being constrained to start after the end of some other activities, such that the makespan of the schedule is minimal. Each activity is executed on one machine, that means it requires one unit of the resource representing machines. The schedule is non-preemptive, i.e. the activities cannot be interrupted once started.

Formally, given a set of $n$ activites $\{A_1, A_2, ..., A_n\}$ with release dates $r_i$, processing times $p_i$ and deadlines $d_i$, a resource $R$ of capacity $m$, unit resource requirement for each activity and a set of precedence constraints $P = \{(A_i, A_j) : A_i \text{ must precede } A_j\}$, produce a schedule $S = (start(A_1), start(A_2), ..., start(A_n))$, such that

\begin{align}
&\text{for all times t} \nonumber\\
&card(\{A_i : start(A_i) \leq t < start(A_i) + p_i\}) \leq m
\end{align}
\begin{align}
\forall_{1 \leq i \leq n} r_i \leq start(A_i)
\end{align}
\begin{align}
\forall_{1 \leq i \leq n} start(A_i) + p_i \leq d_i
\end{align}
\begin{align}
\forall (A_i, A_j) \in P, start(A_i) + p_i \leq start(A_j)
\end{align}
\begin{align}
C_{max} = \max_{1 \leq i \leq n} start(A_i) + p_i
\end{align}

and $C_{max}$ is minimal. For sake of brevity we will use additional variable $end(A_i) = start(A_i) + p_i$. Through the binding to $start(A_i)$ it does not increase complexity and is used strictly for convenience. Additionally we will use $lst(A_i)$ as a latest start time of activity $A_i$ (the latest time $A_i$ can possibly start so it can be finished before its deadline), initially $lst(A_i) = d_i - p_i$. $eet(A_i)$ will denote earliest end time, the earliest possible end time of the activity, initially $eet(A_i) = r_i + p_i$. These times are dynamic, it can turn out there are other activities that have to execute around $r_i$, such that no machine is available. Then $eet(A_i)$ would have to be adjusted and no longer equal to $r_i + p_i$.

In case of problem instances where some deadlines are not specified, it suffices to set them to some high value (end of the schedule horizon). When we describe the method of optimization it will become clear how to find this number.

Note that this formulation is not final, it will be modified in following sections to accommodate uncertainty.

\section{Robustness and uncertainty}
// what is in this section

\subsection{In general}
// why do we bother and how it is done in general

\subsection{In scheduling}
// how can we model uncertainty in scheduling, what is being done in this area

\subsection{In our problem}
// talk about scenario sets for processing times and finish formal problem formulation



\section{Solution - Constraint Programming}
// what is in this section

\subsection{What is Constraint Programming and how does it apply to scheduling?}
// self-descriptive

\subsection{Constraint propagation algorithms for scheduling}
// mention general principles behind propagation algorithms for scheduling and enumerate most popular

\subsection{Precedence Constraint Posting}
// say what it is and how does it help to solve our problem

\subsection{Proposed solution}
// try to sum up everything from previous subsections to clearly describe the solution


\section{Computational results}
// what is in this section

\subsection{Experiment description}
// what kind of experiments are carried

\subsection{Problem instances}
// describe the instances in general

\subsection{Results}
// present the results and comment on them


\section{Conclusions}
// say what was the outcome, what can further be done and throw in a few nice sounding sentences to have a pretty closing


\bibliographystyle{plain}
\bibliography{thesis}


\end{document}